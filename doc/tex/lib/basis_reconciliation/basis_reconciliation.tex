\clearpage
\graphicspath{{./lib/basis_reconciliation/figures/}}
\section{Basis Reconciliation}

\begin{tcolorbox}	
	\begin{tabular}{p{2.75cm} p{0.2cm} p{10.5cm}} 	
		\textbf{Header File}   &:& basis\_reconciliation\_*.h \\
		\textbf{Source File}   &:& basis\_reconciliation\_*.cpp \\
        \textbf{Version}       &:& 20190410 (Andoni Santos)
	\end{tabular}
\end{tcolorbox}

\maketitle
This block is used to reconcile information between two parties in Quantum Key
Distribution, Alice and Bob, by exchanging information about the basis used to
measure the data or the failure to properly measure the qubit.

\subsection*{Signals}

\begin{table}[h]
	%\centering
	\begin{tabular}{|c|l|}
		\hline
		\textbf{Number of Input Signals} & 3 \\ \hline
        \textbf{Type of Input Signals} & TimeContinuousAmplitudeDiscreteReal,
		 \\
		& Binary and TimeDiscreteAmplitudeContinuousReal \\ \hline
    	\textbf{Number of Output Signals} & 3 \\ \hline
        \textbf{Type of Output Signals} & Binary, Binary and TimeDiscreteAmplitudeContinuousReal \\ \hline
	\end{tabular}
	\caption{Binary source signals}
	\label{table:basis_recon_signals}
\end{table}

\subsection*{Input Parameters}

%	\begin{itemize}
%		\item mode\{PseudoRandom\}\linebreak
%		(Random, PseudoRandom, DeterministicCyclic, DeterministicAppendZeros)
%		\item probabilityOfZero\{0.5\}\linebreak
%		(real $\in$ [0,1])
%		\item patternLength\{7\} \linebreak
%		(integer $\in$ [1,32])
%		\item bitStream\{"0100011101010101"\} \linebreak
%		(string of 0's and 1's)
%		\item numberOfBits\{-1\} \linebreak
%		(long int)
%		\item bitPeriod\{1.0/100e9\} \linebreak
%		(double)
%	\end{itemize}

\begin{table}[h]
	\centering
	\begin{tabular}{|c|c|p{40mm}|c|ccp{60mm}}
		\cline{1-4}
		\textbf{Parameter} & \textbf{Type} & \textbf{Values} &   \textbf{Default}& \\ \cline{1-4}
		basisReconciliationRole & basis\_reconciliation\_role & Alice, Bob, Undefined &  Undefined \\ \cline{1-4}
		doubleClickNumber & int & any & 2 \\ \cline{1-4}
		noClickNumber & int &  any & 3 \\ \cline{1-4}
	\end{tabular}
	\caption{Basis reconciliation input parameters}
	\label{table:basis_recon_in_par}
\end{table}

\subsection*{Methods}

\indent BasisReconciliation(initializer\_list<Signal *> InputSig, initializer\_list<Signal *> OutputSig);
\bigbreak	
void initialize(void);
\bigbreak	
bool runBlock(void);
\bigbreak
void setBasisReconciliationRole(basis\_reconciliation\_role role);
\bigbreak
void setDoubleClickNumber(t\_integer dcn);
\bigbreak
void setNoClickNumber(t\_integer ncn);
\bigbreak
long int getReceivedQubits(void);
\bigbreak
long int getReconciledBits(void);

\subsection*{Functional description}

This block has two different behaviors: one for Alice (the transmitter), and
another for Bob (the receiver).
Initially, after Alice encodes the data with the chosen basis and sends it to
Bob, she waits until Bob decides to start the process.

Bob, after successfully receiving data from the quantum channel, sends a message
to Alice with the information required to reconcile the data. This message
contains the either the basis that Bob used to measure the incoming qubits, or,
in case it happens, a code indicating that either the detectors both clicked, or
none of them clicked. After sending the message, Bob stores the data until he
gets a response from Alice.

When Alice receives the message, she compares the basis with the ones she used,
and prepares a response. If the basis were equal, she writes a 1 in the message and outputs that
bit; otherwise, she writes 0 in the message and discards that bit.

Bob then receives the message and, according to the results for each bit, he
either outputs it or discards it. He then waits until he has more data so that
this process can begin again.

\subsection*{Suggestions for future improvement}

Figure out if a minimum number of values should be implemented, in order to
avoid wasting time communicating to reconcile a small number of bits. Typically,
communication using TCP on the classical channel should be minimized, as each
transmission requires an acknowledgement.,The time needed for a round trip is
not negligible, particularly if done frequently.

