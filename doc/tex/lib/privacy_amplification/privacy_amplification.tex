\clearpage
\graphicspath{{./lib/privacy_amplification/figures/}}
\section{Privacy amplification}

\begin{tcolorbox}	
	\begin{tabular}{p{2.75cm} p{0.2cm} p{10.5cm}} 	
        \textbf{Header Files}    &:& privacy\_amplification\_*.h \\
		\textbf{Source Files}    &:& privacy\_amplification\_*.cpp \\
        \textbf{Version}         &:& 20190410 (Andoni Santos)
	\end{tabular}
\end{tcolorbox}

\maketitle
This block implements privacy amplification by hashing segments of the key, and
discarding bits from the corrected key, eliminating the residual knowledge that
an attacker may still have at the end of the error correction process. It
can share a seed between the parties to allow randomly choosing a common hash
family, and may be easily adapted to use a different hash function.


\subsection*{Signals}

If paRole is seedTransmitter:

\begin{table}[h]
	%\centering
	\begin{tabular}{|c|l|}
		\hline
		\textbf{Number of Input Signals} & 2 \\ \hline
        \textbf{Type of Input Signals} & Binary, TimeDiscreteAmplitudeContinuousReal \\ \hline
    	\textbf{Number of Output Signals} & 2 \ \\ \hline
        \textbf{Type of Output Signals} & Binary, TimeDiscreteAmplitudeContinuousReal \\ \hline
	\end{tabular}
	\caption{Privacy amplification signals - Seed Tx}
	\label{table:privacy_amp_tx_signals}
\end{table}

If paRole is seedReceiver:

\begin{table}[h]
	%\centering
	\begin{tabular}{|c|l|}
		\hline
		\textbf{Number of Input Signals} & 3 \\ \hline
        \textbf{Type of Input Signals} & Binary,
        TimeDiscreteAmplitudeContinuousReal,
	   \\
	   & TimeDiscreteAmplitudeContinuousReal \\ \hline
    	\textbf{Number of Output Signals} & 1 \ \\ \hline
        \textbf{Type of Output Signals} & Binary \\ \hline
	\end{tabular}
	\caption{Privacy amplification signals - Seed Rx}
	\label{table:privacy_amp_rx_signals}
\end{table}


\subsection*{Input Parameters}

% %	\begin{itemize}
% %		\item mode\{PseudoRandom\}\linebreak
% %		(Random, PseudoRandom, DeterministicCyclic, DeterministicAppendZeros)
% %		\item probabilityOfZero\{0.5\}\linebreak
% %		(real $\in$ [0,1])
% %		\item patternLength\{7\} \linebreak
% %		(integer $\in$ [1,32])
% %		\item bitStream\{"0100011101010101"\} \linebreak
% %		(string of 0's and 1's)
% %		\item numberOfBits\{-1\} \linebreak
% %		(long int)
% %		\item bitPeriod\{1.0/100e9\} \linebreak
% %		(double)
% %	\end{itemize}

\begin{table}[h]
	\centering
	\begin{tabular}{|c|c|p{40mm}|c|ccp{60mm}}
		\cline{1-4}
		\textbf{Parameter} & \textbf{Type} & \textbf{Values} &   \textbf{Default}& \\ \cline{1-4}
		ber 			& t\_real 		& [0,1] 		& 0.03 \\ \cline{1-4}
		paRole & privacy\_amplification\_role paRole & \{SeedTransmitter,
		SeedReceiver, Undefined\} & Undefined \\ \cline{1-4}
		securityParameter & int &  Any natural number & 1 \\ \cline{1-4}
		bypassHashing & bool & true or false & false \\ \cline{1-4}
	\end{tabular}
	\caption{Privacy amplification input parameters}
	\label{table:bin_sour_in_par}
\end{table}

\subsection*{Methods}

PrivacyAmplification(initializer\_list<Signal*> InputSig,
initializer\_list<Signal*> OutputSig) : Block(InputSig, OutputSig) \{\};
\bigbreak
initialize(void);
\bigbreak
runBlock(void);
\bigbreak
setRole(privacy\_amplification\_role role)
\bigbreak
setBER(t\_real berValue)
\bigbreak
setSystematicSecurityParameter(t\_integer sp)
\bigbreak
setBypassHashing(bool mrb)
\bigbreak
getInputBits(void)
\bigbreak
getOutputBits(void)
\bigbreak

\subsection*{Theoretical introduction}

As mentioned at the beginning, this block is used to amplify privacy on a
quantum key distribution protocol. For instance, after the error correction part
of the BB84 protocol, both parties share equal keys. During the previous steps
of the key distillation process, information about the key was shared in a
public classical channel, but all the shared information has been discarded, and
thus, rendered useless for an attacker. However, the attacker may still have
residual information about the key, acquired during the transmission stage. In
addition, it may be desirable not to discard the bits during error correction,
and deal with privacy issues all in one place.

To understand why a third party (lets call her Eve) may still have information,
recall that they may attempt to intercept the qubits sent by Alice. Let us
assume an intercept-resend arrack. When Eve tries to intercept a certain Qubit,
she sends to Bob the value in the basis she used.

In addition, let us assume that Bob has estimated the QBER to be equal
to a value $z$, and that transmission is perfect. This means that there is no
source of error other and Eve, and thus every error in the received key is
explicitly due to Eve's interference.

There are four possibilities:
\begin{enumerate}
	\item Eve uses the correct basis, and Bob also uses the correct basis. \textit{The
	bit is not discarded, and Eve knows it deterministically}.
	\item Eve uses the correct basis, and Bob uses the wrong basis. The bit is
	discarded, and no-one gains anything.
	\item Eve uses the wrong basis, and Bob uses the wrong basis. The bit is
	discarded, and no-one gains anything.
	\item Eve uses the wrong basis, and Bob uses the correct basis. \textit{The
	bit is not discarded, and it has 50\% chance to introduce and error. Alice
	gains no information from this exchange, other than that she used the wrong
	basis.}
\end{enumerate}

This means that the deterministic knowledge that Eve has on the final key should
be at most twice the QBER. Typically, due to the existence of other sources of
error, the QBER is actually higher than half of Eve's knowledge.

Therefore, the rate of bits she might know is given by the
number of times she measured correctly, which is twice the number of times she
introduced an error, $2z$.

It is important to emphasize that this analysis is based on only one type of
attack. Nevertheless, it is enough to understand the issue. The important fact
is that the information that Eve has is related to the estimated QBER.

One way to remove that information is to use a compression function, which
transforms a key of size $n$ into a key of size $n - t - s$, where $t$ is Eve's
information, and $s$ is a security parameter. This transformation needs to be
random, uniform and unique. One way to do this is with universal hash functions.

\subsection*{Functional description}
At this point it is assumed that Alice and Bob share equal keys. Any error will
be propagated, as we will explain ahead. Therefore, it is crucial that the error
correction is configured with parameters strict enough to avoid failure.

The way this block currently works is that it uses a predefined size between Alice and
Bob, which will be used to hash key segments. By always using segments with the
same size, Alice and Bob are always synchronized and keep equal keys after
privacy amplification.

A seed is generated and shared by the Transmitter. That seed is used to
choose from a universal family of hash functions. The function is then used to
hash the chosen key segment, ending up with a bit string of the same size.
If there is a single bit error in the chosen segment, the hashed string will be
completely different, and will likely have a 50\% error rate (provided that the
hash function is appropriate). We then remove $t+n$ bits from the hashed bit
string, and output the value.

When the receiver block receives the seed it replicates the process, ending up
with the same hashed string (assuming no error is present).


\subsection*{Suggestions for future improvement}

% Currently, the number and type of signals is fixed, and cannot easily be
% adapted. In the future, this could be changed to allow easily adapting the
% message processors to other situations. In order to do that, it should be
% possible to choose a given number of signals and define the desired type for
% each of them.

