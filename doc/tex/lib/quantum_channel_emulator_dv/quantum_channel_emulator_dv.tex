\clearpage
\graphicspath{{./lib/quantum_channel_emulator_dv/figures/}}
\section{Quantum Channel Emulator DV}

\begin{tcolorbox}	
	\begin{tabular}{p{2.75cm} p{0.2cm} p{10.5cm}} 	
        \textbf{Header Files}    &:& quantum\_channel\_emulator\_dv\_*.h \\
		\textbf{Source Files}    &:& quantum\_channel\_emulator\_dv\_*.cpp \\
        \textbf{Version}        &:& 20190328 (Andoni Santos)
	\end{tabular}
\end{tcolorbox}

\maketitle
The Quantum Channel Emulator DV block tries to imitate the possible outcomes
of the discrete variable quantum channel used in the laboratory.
There is another more complex model of this system. However, the additional
complexity makes it difficult to test and troubleshoot the software to be used
with it. Therefore, this block was created. It is able to simulate the
conditions existent on the complex system conditions in a stationary manner.This
way, it produces similar results, while being much faster and simpler to
interact with.

% \begin{multicols}{3}
% \begin{enumerate}
% 	\item Random
% 	\item PseudoRandom
% 	\item DeterministicCyclic
% 	\item DeterministicAppendZeros
%     \item AsciiFileAppendZeros
%     \item AsciiFileCyclic
% \end{enumerate}
% \end{multicols}

\subsection*{Signals}

\begin{table}[h]
	%\centering
	\begin{tabular}{|c|l|}
		\hline
		\textbf{Number of Input Signals} & 4 \\ \hline
        \textbf{Type of Input Signals} & Binary, Binary, Binary, Binary \\ \hline
    	\textbf{Number of Output Signals} & 3 \\ \hline
		\textbf{Type of Output Signals} & TimeContinuousAmplitudeDiscreteReal, \\
		& Binary,TimeContinuousAmplitudeContinuousReal \\ \hline
	\end{tabular}
	\caption{Quantum Channel Emulator signals}
	\label{table:QCE_signals}
\end{table}

\subsection*{Input Signals}

\begin{itemize}
	\item[0] - Binary signal with Alice's key, to be transmitted to Bob.
	\item[1] - Binary signal with Alice's basis, used to encode the key
	\item[2] - Binary signal with transmission mode.
	\item[3] - Binary signal with Bob's basis, used to measure the
	transmitted data.
\end{itemize}

\subsection*{Output Signals}

\begin{itemize}
	\item[0] - TimeContinuousAmplitudeDiscreteReal signal containing the
	results from measuring the transmitted data on Bob's side.
	\item[1] - Binary signal with transmission mode.
	\item[2] - TimeContinuousAmplitudeDiscreteReal signal containing the
	upper bound for the QBER estimate.
\end{itemize}


\subsection*{Input Parameters}

\begin{table}[h]
	\centering
	\begin{tabular}{|c|c|p{20mm}|c|p{35mm}|}
		\hline
		\textbf{Parameter} & \textbf{Type} & \textbf{Values} & \textbf{Default}& \textbf{Description} \\ \hline
		doubleClickNumber & t\_integer & any &  2 & Number to signal that
		both detectors clicked \\\hline
		noClickNumber & t\_integer & any &  3 & Number to signal that
		no detectors clicked \\\hline
		qber & t\_real & [0,1] &  0 & Desired average quantum bit error rate \\\hline
		probabilityOfDoubleClick & t\_real & [0,1] &  0 & Probability of
		both the detectors clicking\\\hline
		probabilityOfNoClick & t\_real & [0,1] &  0 & Probability of
		both no detectors clicking\\\hline
		seed & t\_integer & any & 1 & Random number generator seed.\\\hline
		deadtime & double & any & $1\times10^-5$ & Amount of time the detectors
		aren't responsive after clicking\\\hline
		detectorEfficiency & double & [0,1] & 0.25 & Efficiency of the detectors
		receiving the photons. \\\hline
		fiberAttenuation\_dB & double & $\geq 0$ & 0.2 & Fiber optical
		attenuation in dB \/ km. \\\hline
		fiberLength & double & $\geq 0$ & 50 & Length of the fiber in km. \\\hline
		insertionLosses\_dB & double & $\geq 0$ & 2 & Attenuation by insertion losses. \\\hline
		numberOfPhotonsPerInputPulse & double & $\geq 0$ & 0.1 & Statistical
		average of the number of photons per input pulse. \\\hline
	\end{tabular}
	\caption{Binary source input parameters}
	\label{table:bin_sour_in_par}
\end{table}

\subsection*{Methods}

QuantumChannelEmulatorDv(initializer\_list<Signal *> InputSig, initializer\_list<Signal *> OutputSig) : Block(InputSig, OutputSig) \{\};
\bigbreak	
void initialize(void);
\bigbreak	
bool runBlock(void);
\bigbreak
initialize(void);
\bigbreak
runBlock(void);
\bigbreak
setProbabilityOfBothClick(t\_real pbc)
\bigbreak
setProbabilityOfNoClick(t\_real pnc)
\bigbreak
setProbabilityOfError(t\_real pe)
\bigbreak
setNoClickNumber(t\_integer ncn)
\bigbreak
setDoubleClickNumber(t\_integer dcn)
\bigbreak
setDeadtime(t\_integer dt)
\bigbreak
setDetectorEfficiency(t\_real def)
\bigbreak
setFiberAttenuation\_dB(t\_real fatt)
\bigbreak
setFiberLength(t\_real fl)
\bigbreak
setInsertionLosses\_dB(t\_real il)
\bigbreak
setNumberOfPhotonsPerInputPulse(t\_real nPhotons)
\bigbreak

\subsection*{Functional description}
This block receives three main input signals: the key provided by Alice, the basis
chosen by Alice to encode the key, and the basis chosen by Bob to decode the
key.

The \textit{onDeadtime} and \textit{elapsedDeadtime} variables control whether
the detectors are recovering from a click. Each time anything except a \textit{noClick} is
output, the variable \textit{onDeadtime} is set to true. From them on, on every
iteration, the \textit{symbolPeriod} of the data signal is added to the
\textit{elapsedDeadtime} variable. When that number grows higher than
\textit{deadtime}, the \textit{elapsedDeadtime} is set to zero and
\textit{onDeadtime} is set to false.

The process to determine the output of the Bob's measurement is as follows:

\begin{enumerate}
	\item If \textit{onDeadtime} is true, a \textit{noClickNumber} is output;
	\item A random number between zero and one, \textit{randomRollClicks}, is drawn. If that number is lower
	than \textit{probabilityOfNoClick}, a \textit{noClickNumber} is output;
	\item else, if that number is lower than \textit{probabilityOfNoClick} +
	\textit{probabilityOfDoubleClick}, a \textit{doubleClickNumber} is output;
	\item If none of the above happens, another random number between zero and one, \textit{randomRollError}, is drawn. If the number is
	lower than \textit{probabilityOfError}, the wrong bit is put in the buffer,
	indicating an error (independently of the measurement basis);
	\item If the \textit{randomRollError} is higher than the
	\textit{probabilityOfError}, then the outcome depends on whether Alice's and
	Bob's basis are equal. If they are, the correct bit is output;
	\item If the basis are different, a random number
	\textit{randomRollErrorFromWrongBasis} between zero and one is generated. If
	the value is lower than 0.5, the correct bit is output. Otherwise, the wrong
	bit is output.
\end{enumerate}

In addition, the third input signal is copied to the second output signal, at
the same rate as the key. The third output signal currently is a placeholder and
produces no output.

\subsection*{Suggestions for future improvement}
